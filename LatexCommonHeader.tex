%font stuffs
\usepackage{polyglossia}
% that is only used for hyphenation IDK
\defaultfontfeatures{Renderer=Basic,Ligatures={TeX}}
\setmainfont{CMU Serif}
\newfontfamily{\cyrillicfont}{CMU Serif}
\setsansfont{CMU Serif}
\newfontfamily{\cyrillicfontsf}{CMU Serif}
\setmonofont[RawFeature=-tlig]{DejaVu Sans Mono}
\newfontfamily{\cyrillicfonttt}{DejaVu Sans Mono}

\setdefaultlanguage{russian}
\setotherlanguages{english}

\usepackage{amsthm,amsfonts,amsmath,amssymb,amscd} %maths
\usepackage[style=gost-numeric,sorting=none,%bibiliography tool
            language=auto, autolang=other]{biblatex}

% For picture includes
\usepackage{graphicx}
\usepackage{subcaption}

\DeclareCaptionLabelSeparator{dash}{ -- }
\captionsetup{labelsep=dash}

\usepackage{indentfirst} %Indent first paragraph after section
\usepackage{csquotes} %Language dependent quotation

\usepackage{hyperref}
\hypersetup{
	colorlinks = false,
	urlcolor = blue
}

\usepackage{listings}
\usepackage{xcolor}
\definecolor{Ccomment}{rgb}{0,0.6,0}
\definecolor{Cstring}{HTML}{6A5A3D}
\lstset{basicstyle=
	\footnotesize\selectlanguage{english}\ttfamily,
	tabsize=2,
	texcl=true, % Comments are treated like regular TeX
	commentstyle=\color{Ccomment},
	keywordstyle=\color{brown},
	stringstyle=\color{Cstring},
}

% Environments used
\newtheorem{theorem}{Теорема}
\newtheorem{lemma}{Лемма}
\newtheorem{corollary}{Следствие}

\theoremstyle{definition}
\newtheorem{definition}{Определение}
\newtheorem{assumption}{Предположение}
\newtheorem{example}{Пример}
\newtheorem{statement}{Утверждение}

\let\note\relax
\newtheorem{note}{Замечание}
\newtheorem*{note*}{Замечание}

\newcommand{\td}[1]{\tilde{#1}}
\newcommand{\m}{\phantom{-}}
\newcommand{\sign}{\text{sign}}
