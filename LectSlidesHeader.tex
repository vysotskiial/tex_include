\documentclass[t, pdf, unicode, notheorems]{beamer}

\mode<presentation>
{
%---------------------------тема оформления-------------------------------------------------
    %\usetheme{Madrid}                 %оптимально
    \usetheme{EastLansing}
    \usecolortheme{beaver}

    %\useinnertheme{rectangles}
    \usefonttheme{serif} % <----------- HERE
}
		%\setbeamertemplate{blocks}[default][shadow=false]
\setbeamertemplate{blocks}[shadow=false]
\setbeamertemplate{navigation symbols}{}

\setbeamercolor{block body}{bg=}
		%\setbeamercolor{block title}{bg=,fg=black}


\usepackage{fontenc}     % внутренняя T2A кодировка TeX
\usepackage[utf8]{inputenc}   % кодировка - можно использовать [cp866] [utf8]
\usepackage[russian]{babel}   % включение переносов
\usepackage{dsfont}          % двойные буквы
\usepackage{helvet}
\usepackage{amsmath}
\usepackage{amsfonts}
\usepackage{amssymb}
\usepackage{amsthm}

\usepackage{mdwtab}
\usepackage{graphicx}


%\usepackage{blindtext}
\usepackage{enumitem}
\usepackage{enumerate}
\usepackage{textcomp}

%%%%%%%%% Окружения
%\let\definition\relax\newtheorem{theorem}
\setbeamertemplate{theorems}[numbered]
\setbeamertemplate{section in toc}{\inserttocsectionnumber.~\inserttocsection}

\newtheorem{theorem}{Теорема}
\newtheorem{lemma}{Лемма}
\newtheorem{corollary}{Следствие}
\newtheorem{definition}{Определение}
\newtheorem{example}{Пример}
\newtheorem{assumption}{Предположение}

\newcommand{\exam}{\refstepcounter{ex_no} {{\par\smallskip\bf {Пример }}{\bf{\arabic{lecnum}.\arabic{ex_no}~}}} }
%\newtheorem{lemma}{Лемма}[lecnum]
%\newtheorem{proposition}{Предложение}[lecnum]
%\newtheorem{claim}{Утверждение}[lecnum]
%\newtheorem{corollary}{Следствие}[lecnum]
%\newtheorem{definition}{Определение}[lecnum]
